%*******************************************************
% Abstract
%*******************************************************
\pdfbookmark[0]{Zusammenfassung}{Zusammenfassung}
\chapter*{Zusammenfassung}
Das Projekt befasst sich mit der Entwicklung eines Assistenzsystems zur 
Eintragung wissenschaftlicher Arbeiten. Das Ziel besteht darin, den 
häufig fehleranfälligen und zeitaufwendigen Umgang mit BibTeX-Dateien 
zu vereinfachen und die Übertragung in das OPUS-Repositorium zu erleichtern. 
Das entwickelte Tool ermöglicht den Import von BibTeX-Dateien, die Bearbeitung 
einzelner Einträge, Such- und Löschfunktionen sowie den Export in das 
OPUS-XML-Format. Die Umsetzung basiert auf einer leichtgewichtigen 
Architektur mit den Technologien React, Node.js und Local Storage. 
Ein durchgeführter Usability-Test bestätigte die Benutzerfreundlichkeit 
und Funktionalität des Systems. Gleichzeitig ergaben sich Hinweise auf 
sinnvolle Erweiterungen, beispielsweise Mehrsprachigkeit, ein Dark Mode 
und die Einführung von Benutzerkonten. Damit bietet das Projekt eine 
praktikable Grundlage zur Vereinfachung des Publikationsprozesses und 
eröffnet Potenzial für künftige Weiterentwicklungen.
