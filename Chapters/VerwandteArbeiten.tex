\chapter{Verwandte Arbeiten}
In diesem Kapitel werden drei bestehende Systeme und Forschungsarbeiten 
vorgestellt, die thematisch eng mit unserem Projekt verbunden sind. 
Ein Vergleich mit Lösungen wie CiteAssist, BIBTEXManager und CRIS-Systemen 
zeigt, in welchem Umfeld sich unser Projekt bewegt, welche Gemeinsamkeiten 
es gibt und worin sich unser Ansatz unterscheidet.

\section{CiteAssist}
Ein besonders relevantes System im Bereich der Literaturverwaltung ist die Arbeit von  
Kaesberg et al. \cite{kaesberg2024citeassist}. 
Es wurde entwickelt, um Preprints mit korrekten und sofort 
zugänglichen Zitationsinformationen auszustatten. Dazu werden 
Metadaten wie Autor, Titel, Jahr oder DOI automatisch aus dem 
Dokument extrahiert und als standardisierter BibTeX-Eintrag 
sowohl am Anfang als auch am Ende des PDFs integriert. Auf 
diese Weise soll die Zitierbarkeit von Preprints verbessert 
werden, da Forschende die notwendigen Informationen nicht 
mehr extern suchen müssen. Darüber hinaus schlägt CiteAssist 
anhand von Schlüsselwörtern verwandte Arbeiten vor und 
erweitert damit die klassische Related-Work-Sektion. 
Gemeinsam mit unserem Projekt verfolgt es das Ziel, 
den Prozess der Literaturverwaltung effizienter und 
weniger fehleranfällig zu gestalten. Während CiteAssist 
jedoch den Schwerpunkt auf die Sichtbarkeit und Verknüpfung 
von Preprints legt, richtet sich unser System auf die 
benutzerfreundliche Bearbeitung von BibTeX-Dateien und 
die direkte Integration in das institutionelle Repositorium 
OPUS. Dadurch unterscheiden sich beide Ansätze in ihrer 
Zielgruppe, ergänzen sich jedoch in ihrer gemeinsamen Intention, 
den Publikationsprozess durch technische Unterstützung zu verbessern.

\section{BIBTEXManager}

Ein anderes  verwandtes Projekt ist die Masterarbeit von Mitesh Pravin Furia \cite{furiabibtex}, in der ein 
System zur Verwaltung und Bearbeitung von Bib\TeX{}-Dateien beschreibt. 
Das von ihm entwickelte Tool ermöglicht den Import von Bib\TeX{}-Einträgen, 
deren Bearbeitung über eine grafische Benutzeroberfläche sowie das Exportieren 
in verschiedenen Formaten. Ziel ist es, die Konsistenz wissenschaftlicher 
Metadaten zu erhöhen und Forschenden eine effiziente Verwaltung ihrer 
Publikationen zu ermöglichen.\\

\noindent Unser Assistenzsystem verfolgt ein ähnliches Ziel, indem es ebenfalls 
Bib\TeX{}-Dateien als Eingangspunkt nutzt. Im Unterschied dazu liegt der 
Fokus unserer Arbeit jedoch auf der Integration in institutionelle 
Repositorien, insbesondere durch den Export der Daten in das von OPUS 
erwartete XML-Format. Außerdem adressieren beide Systeme die zentrale Herausforderung der
effizienten Verwaltung wissenschaftlicher Literaturangaben durch Import-Bearbeitungs und Export-Workflows.
Furia exportiert  hauptsächlich in bibliographische Formate (HTML, RTF, XML).
Seine Lösung erfordert MySQL-Datenbankinstallation für Kollaborationsfeatures, unser System arbeitet dagegen dateibasiert
ohne externe Infrastruktur. Furias System eignet sich gut für Forschungsgruppen, die zusammenarbeiten und verschiedene Ausgabeformate benötigen.
Unser System ist dagegen direkt auf OPUS-Repositories zugeschnitten. Der Vorteil unseres Ansatzes liegt in der einfacheren
Installation ohne Datenbankserver und der automatischen XML-Erzeugung für OPUS. Furias System bietet mehr Funktionen. 
Unser System ist dagegen  gezielt auf universitäre Publikationsprozesse ausgerichtet."

\section{CRIS systems}
Ein weiteres relevantes Beispiel ist das von Kovacevic et al. 
\cite{kovavcevic2011automatic} vorgestellte Framework zur automatisierten 
Extraktion bibliographischer Metadaten aus wissenschaftlichen Artikeln. 
Die vorliegende Arbeit fußt auf der Beobachtung, dass die manuelle 
Erfassung von Literaturangaben mit einem hohen Zeitaufwand verbunden 
ist und zu Fehlern neigt, insbesondere bei der Gewinnung von 
Informationen aus PDF-Dokumenten. Das Framework bedient sich 
einer Kombination aus regelbasierten Verfahren und statistischen 
Methoden, um Metadaten wie Autor, Titel, Publikationsjahr oder 
Journalname direkt aus Volltexten zu extrahieren und in standardisierte 
Formate wie XML oder BibTeX zu überführen. Das Ziel besteht darin, den 
Aufwand für Forschende zu verringern und zugleich die Qualität der 
erzeugten bibliographischen Daten zu erhöhen. Im Rahmen der 
Gegenüberstellung mit unserem Assistenzsystem werden sowohl 
Gemeinsamkeiten als auch Unterschiede evident. Die beiden Ansätze 
verfolgen das Ziel, bibliographische Prozesse effizienter zu gestalten 
und die Fehlerquote bei der Verwaltung wissenschaftlicher Publikationen 
zu senken. Das von uns entwickelte Projekt fokussiert sich auf die 
benutzerfreundliche Bearbeitung von BibTeX-Dateien und die direkte 
Integration in das institutionelle Repositorium OPUS. Das Framework 
konzentriert sich stärker auf die technische Extraktion von Daten aus 
komplexen Layouts wissenschaftlicher Artikel. Diese Unterschiede in der 
Zielgruppe und dem Einsatzkontext machen die Arbeit dennoch für unser 
Projekt relevant, da sie den hohen Bedarf an automatisierten Verfahren 
zur Metadatenverarbeitung aufzeigt und die Komplementarität unterschiedlicher 
technischer Ansätze demonstriert.