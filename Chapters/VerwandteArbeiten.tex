\chapter{Verwandte Arbeiten}

\section{CiteAssist}
Ein besonders relevantes System im Bereich der Literaturverwaltung ist die Arbeit von  
Kaesberg et al. \cite{kaesberg2024citeassist}. 
Es wurde entwickelt, um Preprints mit korrekten und sofort 
zugänglichen Zitationsinformationen auszustatten. Dazu werden 
Metadaten wie Autor, Titel, Jahr oder DOI automatisch aus dem 
Dokument extrahiert und als standardisierter BibTeX-Eintrag 
sowohl am Anfang als auch am Ende des PDFs integriert. Auf 
diese Weise soll die Zitierbarkeit von Preprints verbessert 
werden, da Forschende die notwendigen Informationen nicht 
mehr extern suchen müssen. Darüber hinaus schlägt CiteAssist 
anhand von Schlüsselwörtern verwandte Arbeiten vor und 
erweitert damit die klassische Related-Work-Sektion. 
Gemeinsam mit unserem Projekt verfolgt es das Ziel, 
den Prozess der Literaturverwaltung effizienter und 
weniger fehleranfällig zu gestalten. Während CiteAssist 
jedoch den Schwerpunkt auf die Sichtbarkeit und Verknüpfung 
von Preprints legt, richtet sich unser System auf die 
benutzerfreundliche Bearbeitung von BibTeX-Dateien und 
die direkte Integration in das institutionelle Repositorium 
OPUS. Dadurch unterscheiden sich beide Ansätze in ihrer 
Zielgruppe, ergänzen sich jedoch in ihrer gemeinsamen Intention, 
den Publikationsprozess durch technische Unterstützung zu verbessern.