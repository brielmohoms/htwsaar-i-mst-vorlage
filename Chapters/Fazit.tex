\chapter{Fazit und Ausblick}

\section{Fazit}
Das entwickelte Assistenzsystem zur Eintragung wissenschaftlicher Arbeiten 
erfüllt die wesentlichen Projektziele. Nutzer:innen können BibTeX-Dateien 
importieren, deren Einträge bei Bedarf bearbeiten oder löschen, durchsuchen 
und im Anschluss im OPUS-Format exportieren. Dadurch wird der 
Publikationsprozess deutlich vereinfacht, da wiederkehrende manuelle 
Eingaben und fehleranfällige Formatierungen entfallen.\\

\noindent Hervorzuheben ist zudem die benutzerfreundliche Oberfläche, die eine intuitive 
Interaktion ermöglicht und durch Funktionen wie Echtzeit-Vorschau, Paginierung 
und Suchfilter ergänzt wird. Die positiven Ergebnisse des Usability-Tests 
bestätigen, dass das System in seiner jetzigen Form zuverlässig funktioniert 
und von erfahrenen Anwender:innen als nützlich und effizient wahrgenommen wird.\\

\noindent Gleichzeitig hat das Projekt gezeigt, dass die klare Trennung von Frontend, 
Backend und Datenspeicherung in einer leichtgewichtigen Architektur (React, 
Node.js, Local Storage) eine gute Basis für Erweiterungen bietet. Trotz 
technischer Herausforderungen, insbesondere beim Parsen komplexer 
BibTeX-Dateien und der dynamischen Bearbeitung von Autoren, konnte eine 
stabile Lösung entwickelt werden.

\section{Ausblick}