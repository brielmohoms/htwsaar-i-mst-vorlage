\chapter{Fazit und Ausblick}

\section{Fazit}
Das entwickelte Assistenzsystem zur Eintragung wissenschaftlicher Arbeiten 
erfüllt die wesentlichen Projektziele. Nutzer:innen können BibTeX-Dateien 
importieren, deren Einträge bei Bedarf bearbeiten oder löschen, durchsuchen 
und im Anschluss im OPUS-Format exportieren. Dadurch wird der 
Publikationsprozess deutlich vereinfacht, da wiederkehrende manuelle 
Eingaben und fehleranfällige Formatierungen entfallen.\\

\noindent Hervorzuheben ist zudem die benutzerfreundliche Oberfläche, die eine intuitive 
Interaktion ermöglicht und durch Funktionen wie Echtzeit-Vorschau, Paginierung 
und Suchfilter ergänzt wird. Die positiven Ergebnisse des Usability-Tests 
bestätigen, dass das System in seiner jetzigen Form zuverlässig funktioniert 
und von erfahrenen Anwender:innen als nützlich und effizient wahrgenommen wird.\\

\noindent Gleichzeitig hat das Projekt gezeigt, dass die klare Trennung von Frontend, 
Backend und Datenspeicherung in einer leichtgewichtigen Architektur (React, 
Node.js, Local Storage) eine gute Basis für Erweiterungen bietet. Trotz 
technischer Herausforderungen, insbesondere beim Parsen komplexer 
BibTeX-Dateien und der dynamischen Bearbeitung von Autoren, konnte eine 
stabile Lösung entwickelt werden.

\section{Ausblick}
Für zukünftige Entwicklungen bestehen mehrere sinnvolle 
Erweiterungsmöglichkeiten. Zunächst wäre eine Mehrsprachigkeit der 
Benutzeroberfläche wünschenswert, um eine breitere Nutzergruppe zu erreichen 
und die internationale Verwendbarkeit zu steigern. Die Implementierung von 
Benutzerkonten könnte eine Personalisierung der Nutzung bewirken und die 
Förderung einer kollaborativen Arbeitsweise unterstützen.\\

\noindent Darüber hinaus wird die Integration eines Dark Modes empfohlen, um die 
Benutzerfreundlichkeit weiter zu erhöhen. Auf technischer Ebene besteht 
hinsichtlich des Local Storage Optimierungspotenzial, beispielsweise durch 
den Einsatz optimierter Speicherstrategien, automatisches Zwischenspeichern 
oder Versionierung. Diese Lösung wurde bewusst so konzipiert, dass sie eine 
geringe Größe aufweist und unabhängig von externer Infrastruktur ist. 
Dadurch eignet sie sich in besonderer Weise für den Einsatz in 
Hochschulumgebungen.\\

\noindent Langfristig ließe sich das System von einem reinen Transfer-Tool zu einer 
vollwertigen Publikationsplattform weiterentwickeln. Funktionen wie erweiterte 
Such- und Filteroptionen, Mehrbenutzerunterstützung oder die direkte 
Online-Veröffentlichung würden den Mehrwert für Forschende und Studierende 
nochmals deutlich steigern.