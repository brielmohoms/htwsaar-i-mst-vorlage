\chapter{Diskussion}

Das entwickelte Assistenzsystem zur Übertragung wissenschaftlicher Arbeiten in das OPUS-Publikationssystem erfüllt die wesentlichen 
Zielsetzungen des Projekts. Professoren und Professorinnen haben die Möglichkeit, BibTeX-Dateien zu importieren, die enthaltenen 
Einträge zu bearbeiten oder zu löschen und im Anschluss im OPUS-Format zu exportieren. Die Bedienung erfolgt vollständig über 
eine Weboberfläche, was eine einfache und intuitive Nutzung ermöglicht. Durch den gewählten Technologie-Stack aus React, Node.js 
und Local Storage konnte ein leichtgewichtiger Prototyp realisiert werden, der ohne zusätzliche Server- oder Datenbankinfrastruktur 
auskommt.

Die im Projekt definierten technischen Ziele wurden weitgehend erreicht. Sämtliche Kernfunktionen, nämlich Import, Bearbeitung, 
Suche, Speicherung, Löschung und Export von BibTeX-Daten, sind implementiert und funktionieren zuverlässig innerhalb der gewählten 
Systemarchitektur. Im Zuge der Optimierung der Benutzerfreundlichkeit wurden ergänzend Funktionen wie eine Echtzeitvorschau der 
Einträge und eine automatische Trennung einzelner Felder realisiert. Die vorliegende Umsetzung hat das Potenzial, den 
Publikationsprozess zu erleichtern und die mit manueller Eingabe und Formatierung häufig auftretenden Fehler zu reduzieren.

Während des Implementierungsprozesses traten jedoch verschiedene Herausforderungen auf. Die Analyse der vorliegenden Daten ergab, 
dass das Parsen von .bib-Dateien im Backend mit einer hohen Komplexität verbunden ist. Dies ist auf die Berücksichtigung 
unterschiedlicher Formatvarianten und Sonderzeichen zurückzuführen. Auch die dynamische Bearbeitung einzelner Autoren im 
Frontend stellte eine technische Herausforderung dar, da hierfür eine flexible und gleichzeitig stabile Datenstruktur 
erforderlich war. Darüber hinaus waren organisatorische Schwierigkeiten bei der parallelen Arbeit an Frontend und 
Backend zu verzeichnen, was eine präzise Abstimmung im Team erforderlich machte. Die zuvor genannten Probleme konnten 
durch spezifische Anpassungen in der Parserlogik, erweiterte Validierungsmechanismen und transparente Kommunikationswege 
innerhalb des Teams weitgehend gelöst werden.

Insgesamt hat sich die gewählte Architektur als zweckmäßig erwiesen, um die gestellten Anforderungen mit überschaubarem 
Entwicklungsaufwand umzusetzen. Der modulare Aufbau mit klarer Trennung von Frontend, Backend und Datenspeicherung 
ermöglicht eine einfache Erweiterbarkeit, beispielsweise durch die Integration einer Benutzeranmeldung, 
von Mehrsprachigkeit oder durch die direkte Anbindung an OPUS über eine API. Die Entwicklung hat zudem gezeigt, 
dass ein enger Austausch im Team und eine klare Aufgabenteilung für die termingerechte Umsetzung entscheidend waren.

Für eine zukünftige Weiterentwicklung wäre es sinnvoll, neben der Verbesserung der Datenpersistenz auch Funktionen 
wie Mehrbenutzerunterstützung, erweiterte Suchfilter oder eine direkte Online-Veröffentlichung zu integrieren. 
Durch diese Erweiterungen würde sich das System von einem reinen Übertragungswerkzeug zu einer vollwertigen 
Publikationsplattform entwickeln und den Mehrwert für die Zielgruppe nochmals deutlich steigern.
