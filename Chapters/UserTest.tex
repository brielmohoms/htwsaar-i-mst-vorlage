\chapter{User Test}
Dieses Kapitel beschreibt den durchgeführten User Test, der dazu dient ,
die Benutzerfreundlichkeit und Funktionalität der entwickelten Anwendung praxisnah und unter
realitätsnahen Bedingungen zu evaluieren.

\section{Methode und Ablauf}
Der User Test wurde mit drei Master-Studierenden der Informatik durchgeführt, 
die bereits Erfahrung im Umgang mit BibTeX-Dateien hatten. Die Testpersonen führten verschiedene Aufgaben aus,
darunter das Importieren einer BibTeX-Datei, das Bearbeiten und Löschen von Einträgen,
die Nutzung der Suchfunktion ,das Speichern der Änderungen, die Navigation  durch die Einträge mittels Paginierung 
sowie das Exportieren der Daten.
Die Anwendung wurde lokal ausgeführt, da sie noch nicht deployed war. Der Test fand in einem Hörsaal
der Hochschule in einer Live-Umgebung statt, um eine realistische Nutzungssituation zu simulieren.

\section{Ergebnisse}
Die Testpersonen konnten die Funktionen erfolgreich ausführen und gaben wertvolles Feedback zur
Benutzerfreundlichkeit und Performance der Anwendung. Insbesondere wurden die Import-Export Funktion und
die Bearbeitungsmöglichkeiten als intuitiv bewertet. Die Paginierung wurde als hilfreich empfunden, um den 
Überblick bei großen Datensätzen zu behalten.

Als Verbesserungsvorschläge wurde genannt, dass ein Dunkelmodus (Dark Mode) gewünscht wird, 
da die Anwendung derzeit nur in hellem Design verfügbar ist. Außerdem wurde angemerkt, 
dass die Schaltflächen „Eintrag löschen“ und „Alle Einträge löschen“ auf den ersten Blick nicht klar voneinander 
zu unterscheiden sind.Zusätzlich wurde von den Testpersonen der Wunsch geäußert, die Anwendung mehrsprachig anzubieten,
um eine breitere Nutzergruppe anzusprechen und die Bedienbarkeit für Anwender mit unterschiedlichem sprachlichem Hintergrund
zu verbessern. Insgesamt bestätigte der User Test die Grundfunktionalität und gab Hinweise für zukünftige Verbesserungen.

\section{Interpretation der Ergebnisse und deren Auswirkungen}
Die erfolgreichen Tests bestätigen, dass die Grundfunktionalität der Anwendung den Anforderungen der Zielgruppe entspricht
und die Kernprozesse wie Import, Bearbeitung, Suche und Navigation , Export zuverlässig funktionieren. Die positive Rückmeldung zur
Benutzerfreundlichkeit und Performance zeigt, dass das System gut auf die Bedürfnisse erfahrener Nutzer abgestimmt ist.

Die Hinweise auf den Wunsch nach einem Dunkelmodus und die Unklarheit bei den Lösch-Schaltflächen verdeutlichen,
dass trotz funktionaler Stabilität das Nutzererlebnis weiter verbessert werden kann. Insbesondere die visuelle 
Gestaltung und Bedienbarkeit spielen eine wichtige Rolle für die Akzeptanz der Anwendung im Alltag.

Diese Erkenntnisse haben direkte Auswirkungen auf die weitere Entwicklung: Die Implementierung eines Dark Modes
sowie eine klarere Differenzierung der Bedienelemente sollten prioritär umgesetzt werden, um die Usability weiter 
zu steigern. Darüber hinaus legen die Ergebnisse nahe, dass die Einführung einer mehrsprachigen Benutzeroberfläche
von großem Vorteil wäre, um die Zugänglichkeit und Nutzerfreundlichkeit für Anwender mit unterschiedlichem sprachlichen
Hintergrund zu erhöhen. Dies könnte die Anwendung für ein internationales Publikum öffnen und ihre Verbreitung fördern.
Zukünftige Tests mit einer größeren und vielfältigeren Nutzergruppe erscheinen ebenfalls sinnvoll,
um weitere Optimierungspotenziale zu identifizieren und die Anwendung noch breiter nutzbar zu machen.

Insgesamt unterstreichen die Resultate die Bedeutung praxisnaher Tests frühzeitig im Entwicklungsprozess,
um gezielt auf Benutzerbedürfnisse einzugehen und die Softwarequalität nachhaltig zu verbessern.