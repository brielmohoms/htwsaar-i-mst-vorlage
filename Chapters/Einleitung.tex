\chapter{Einleitung}
In diesem Kapitel wird der Projekt vorgestellt. Es wird die Motivation 
hinter der Entwicklung beschrieben, das zugrundeliegende Problem erläutert
und schließlich das Ziel des Projekts klar formuliert. 

\section{Motivation}
Für wissenschaftliche Arbeiten ist eine saubere und konsistente Verwaltung 
von Literaturquellen unerlässlich. BibTeX-Dateien werden oft manuell 
bearbeitet, was schnell zu Formatierungsfehlern oder unübersichtlichen 
Einträgen führen kann. Für Nutzer ist es weder einfach noch bequem, 
alle Felder einer .bib-Datei manuell und korrekt auszufüllen, besonders 
wenn es viele Einträge gibt. Deshalb haben wir ein benutzerfreundliches 
Assistenzsystem zur Eintragung wissenschaftlicher Arbeiten entwickelt, 
um Professoren die Arbeit zu erleichtern.

\section{Problembeschreibung}


\section{Ziel des Projekts}
Das Ziel des Projekts ist die Entwicklung eines einfach zu bedienenden 
Tools zur Erfassung und Verwaltung wissenschaftlicher Arbeiten. 
Nutzerinnen und Nutzer sollen die Möglichkeit haben, eine bestehende 
.bib-Datei auszuwählen, deren Einträge sich einzeln anzeigen und bei 
Bedarf bearbeiten. Ein weiterer zentraler Bestandteil des Tools ist die direkte Exportfunktion
in das Repositorium Opus. Damit wird der Publikationsprozess erheblich 
vereinfacht und beschleunigt. Darüber hinaus trägt das System dazu bei, 
die Qualität und Einheitlichkeit der bibliographischen Daten zu verbessern,
indem Fehlerquellen minimiert und wiederkehrende Arbeitsschritte 
automatisiert werden. Langfristig ermöglicht das Tool eine bessere Organisation 
wissenschaftlicher Arbeiten und eine Zeitersparnis für die 
Nutzerinnen und Nutzer, wodurch der gesamte Archivierungs- 
und Publikationsprozess effizienter gestaltet wird.

