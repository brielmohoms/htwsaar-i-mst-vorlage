\chapter{Einleitung}
In diesem Kapitel wird das Projekt vorgestellt. Es wird die Motivation 
hinter der Entwicklung beschrieben, das zugrundeliegende Problem erläutert
und schließlich das Ziel des Projekts klar formuliert. 

\section{Motivation}
Die Idee für dieses Projekt entstand aus der Beobachtung, dass viele 
Professorinnen und Professoren beim Archivieren ihrer wissenschaftlichen 
Publikationen in OPUS\footnote{OPUS. Verfügbar unter: 
\url{https://www.opus-repository.org/userdoc/} (zugegriffen August 2025).} 
häufig auf unnötige Hürden stoßen. Der aktuelle Prozess 
erfordert oft eine manuelle Eingabe der Daten, was sowohl zeitintensiv 
als auch fehleranfällig ist. Besonders bei umfangreichen Publikationslisten 
kann dies zu einem erheblichen Aufwand führen.\\

\noindent Das Projekt verfolgt daher die Idee, ein einfaches und 
benutzerfreundliches Tool zu entwickeln. Mit diesem sollen Nutzerinnen und 
Nutzer eine \texttt{.bib}-Datei – ein gängiges Format zur Speicherung 
bibliographischer Daten – importieren können. Anschließend sollen sie die 
Möglichkeit haben, die Einträge einzeln einzusehen und bei Bedarf zu 
bearbeiten. Auf diese Weise wird der gesamte Vorgang transparenter und 
effizienter gestaltet.\\

\noindent Die Relevanz dieses Ansatzes liegt in der deutlichen Zeitersparnis und der 
Verbesserung der Datenqualität. Durch die Möglichkeit, die aufbereiteten 
Einträge direkt in OPUS zu exportieren, wird der Publikationsprozess 
erheblich vereinfacht. Dies führt nicht nur zu einer besseren Organisation 
wissenschaftlicher Arbeiten, sondern unterstützt auch die Hochschulen bei 
der langfristigen und fehlerfreien Archivierung von Forschungsleistungen.

\section{Problembeschreibung}
In dem  heutigen wissenschaftlichen Bereich spielt die Verwaltung von Literaturquelle eine große Rolle.
In diesem Zusammenhang wird zur Verwaltung und Zitierung der Literaturquellen eine \texttt{.bib}-Datei verwendet,
die eine strukturierte und automatisierte Einbindung der Referenzen im Dokument ermöglicht.
Ein häufiges Problem besteht darin, dass die manuelle Bearbeitung von \texttt{.bib}-Datei zu 
Formatierungsfehlern oder sogar zu fehlenden Einträgen in der Literaturverzeichnis führen kann.
Besonders bei umfangreichen Projekten mit vielen Quellen wird die Pflege der Datei schnell unübersichtlich und fehleranfällig.
Auch kleinere Eingabefehler können dazu führen, dass Quellen nicht korrekt angezeigt oder überhaupt nicht übernommen werden.
Deshalb haben wir ein ein benutzerfreundliches Assistenzsystem zur Eintragung wissenschaftlicher Arbeiten entwickelt, um dieses Problem zu lösen.
Dieses System soll die Qualität der Literaturverwaltung verbessern und den Arbeitsaufwand für die Professoren und 
Studierenden deutlich reduzieren.


\section{Ziel des Projekts}
Das Ziel des Projekts ist die Entwicklung eines einfach zu bedienenden 
Tools zur Erfassung und Verwaltung wissenschaftlicher Arbeiten. 
Nutzerinnen und Nutzer sollen die Möglichkeit haben, eine bestehende 
.bib-Datei auszuwählen, deren Einträge sich einzeln anzeigen und bei 
Bedarf bearbeiten. Ein weiterer zentraler Bestandteil des Tools ist die direkte Exportfunktion
in das Repositorium OPUS. Damit wird der Publikationsprozess erheblich 
vereinfacht und beschleunigt. Darüber hinaus trägt das System dazu bei, 
die Qualität und Einheitlichkeit der bibliographischen Daten zu verbessern,
indem Fehlerquellen minimiert und wiederkehrende Arbeitsschritte 
automatisiert werden. Langfristig ermöglicht das Tool eine bessere Organisation 
wissenschaftlicher Arbeiten und eine Zeitersparnis für die 
Nutzerinnen und Nutzer, wodurch der gesamte Archivierungs- und 
Publikationsprozess effizienter gestaltet wird.

