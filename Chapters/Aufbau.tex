\chapter{Aufbau des Projektteams}
In diesem Kapitel wird beschrieben, wie das Projektteam war und wie die 
Zusammenarbeit im Verlauf des Projekts war. Zunächst wird ein Überblick über
das Team gegeben, anschließend wird erläutert, wie die Arbeitsaufteilung
und Kommunkition erfolgte. Am Ende werden die einzelnen Rollen und Aufgaben
der Teammitglieder im dargestellt

\section{Übersicht}

\section{Plan zur Zusammenarbeit im Team}


Zu Beginn bestand unser Team aus drei Personen, allerdings musste eine Person das Projekt leider verlassen.
Trotz dieser Veränderung haben wir die Zusammenarbeit effektiv organisiert.
Wir haben Git als Versionskontrollsystem genutzt, um den Code gemeinsam zu verwalten und Konflikte zu vermeiden.
Regelmäßig fanden Meetings mit dem Professor  statt, um den Fortschritt zu besprechen
und Rückmeldungen einzuholen. Nach jedem Meeting mit dem Professor trafen wir uns im Team, um die Aufgaben zu verteilen
und den weiteren Arbeitsplan zu besprechen.
Jede Person arbeitete eigenständig an ihren zugeteilten Aufgaben.
Dabei gaben wir uns gegenseitig Feedback und konstruktive Kritik,
um die Qualität unserer Arbeit kontinuierlich zu verbessern. 
Dieses Vorgehen haben wir bis zum aktuellen Stand des Projekts beibehalten.

Bezüglich der Implementierung waren ursprünglich zwei Personen für das Backend zuständig,
insbesondere für den Parser-Service und die API-Endpunkte, während eine Person
das Frontend entwickelte. Nachdem eine Person das Team verlassen hatte,
haben die verbleibenden zwei Teammitglieder die Arbeit gerecht untereinander aufgeteilt, um alle Aufgaben weiterhin abzudecken.



\section{Funktion der einzelnen Personen und Umsetzungskette}