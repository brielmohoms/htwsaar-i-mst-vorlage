\chapter{Aufbau des Projektteams}
In diesem Kapitel wird beschrieben, wie das Projektteam zusammengesetzt war 
und wie die Zusammenarbeit im Verlauf des Projekts verlief. Zunächst 
wird das Team vorgestellt, anschließend wird erläutert, wie die 
Arbeitsaufteilung und Kommunikation erfolgte. Zum Schluss erfolgt eine 
Darstellung der einzelnen Rollen und Aufgaben der Teammitglieder.

\section{Übersicht}
Das Projekt wurde ursprunglich in einer Dreiergruppe gestartet durchgeführt, 
bestehend aus Briel Mohomye, Vanelle Tchinda und Tony Nsangou. Aufgrund 
organisatorischer Schwierigkeiten konnte das dritte Teammitglied jedoch 
nicht offiziell am Projekt teilnehmen. Aus diesem Grund wurde das Projekt 
im weiteren Verlauf von Briel Mohomye und Vanelle Tchinda eigenständig 
durchgeführt. Beide Teammitglieder haben gleichberechtigt an der Konzeption, 
Entwicklung und Dokumentation des Tools mitgewirkt. Ziel war es, 
die Stärken und Kompetenzen beider Personen optimal einzubringen, 
um die Arbeit effizient und zielgerichtet zu gestalten.

\section{Plan zur Zusammenarbeit im Team}
Wir haben Git als Versionskontrollsystem genutzt, um den Code gemeinsam zu
verwalten und Konflikte zu vermeiden. Regelmäßig fanden Meetings mit dem Professor  statt, 
um den Fortschritt zu besprechen und Rückmeldungen einzuholen. Nach jedem 
Meeting mit dem Professor trafen wir uns im Team, um die Aufgaben zu 
verteilen und den weiteren Arbeitsplan zu besprechen.
Jede Person arbeitete eigenständig an ihren zugeteilten Aufgaben.
Dabei gaben wir uns gegenseitig Feedback und konstruktive Kritik,
um die Qualität unserer Arbeit kontinuierlich zu verbessern. 
Dieses Vorgehen haben wir bis zum aktuellen Stand des Projekts beibehalten.

Bezüglich der Implementierung waren ursprünglich zwei Personen für das Backend zuständig,
insbesondere für den Parser-Service und die API-Endpunkte, während eine Person
das Frontend entwickelte. Nachdem eine Person das Team verlassen hatte,
haben die verbleibenden zwei Teammitglieder die Arbeit gerecht untereinander aufgeteilt, 
um alle Aufgaben weiterhin abzudecken.


\section{Funktion der einzelnen Personen und Umsetzungskette}
Zu Beginn der technischen Umsetzung konzentrierten sich zwei Teammitglieder auf die Backend-Entwicklung,
mit dem Fokus auf den Aufbau eines Parser-Services zur Konvertierung von \texttt{.bib}-Dateien in
ein JSON-Format sowie auf die Gestaltung der benötigten API-Endpunkte.
Parallel dazu begann ein drittes Teammitglied mit der Gestaltung des Frontends.
Nach dem vorzeitigen Ausscheiden eines Teammitglieds übernahmen die
beiden verbleibenden Personen zusätzlich die Weiterentwicklung der Benutzeroberfläche und
stimmten ihre Aufgaben fortan eng miteinander ab.
Die Umsetzung begann mit der Entwicklung des Parsers im Backend,
welcher die Einträge aus der \texttt{.bib}-Datei in ein JSON-Format überträgt.
Darauf folgte die Erstellung der API-Endpunkte zur Kommunikation mit dem Frontend.

Im Anschluss wurde mit der Frontend-Entwicklung begonnen. Dabei wurde schrittweise folgendes umgesetzt:
\begin{itemize}
    \item Implementierung eines \textbf{„Importieren“-Buttons} zur Auswahl einer \texttt{.bib}-Datei.
    \item Darstellung der Einträge in einer \textbf{aufklappbaren Struktur}.
    \item Möglichkeit zur \textbf{Bearbeitung einzelner Einträge} direkt in der Oberfläche.
    \item Einbindung einer \textbf{Suchkomponente} zur Filterung der Einträge.
    \item Implementierung der \textbf{Paginierung} zur besseren Übersicht bei vielen Einträgen.
    \item Nutzung von \textbf{Local Storage}, um Änderungen lokal zu speichern.
    \item Einfügen eines \textbf{„Speichern“-Buttons} zur endgültigen Sicherung der bearbeiteten Daten.
    \item Trennung der \textbf{Autoreninformationen} aus der \texttt{.bib}-Datei in einzelne Felder.
    \item Einfügen eines \textbf{Buttons zum Löschen aller Einträge}.
    \item Hinzufügen eines \textbf{„Exportieren“-Buttons} zur Ausgabe der modifizierten Daten.
\end{itemize}

Im Verlauf des Projekts wurden die einzelnen Entwicklungsschritte regelmäßig
auf Grundlage der Besprechungen mit dem Betreuer angepasst. Dabei wurde großer Wert auf transparente
Kommunikation und iterative Verbesserung gelegt. Durch kontinuierliche Rücksprachen und konstruktiven Austausch 
konnte die Arbeit effizient koordiniert und gemeinsam vorangetrieben werden.