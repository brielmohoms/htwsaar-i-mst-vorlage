\chapter{Aufbau des Projektteams}
In diesem Kapitel wird beschrieben, wie das Projektteam zusammengesetzt war 
und wie die Zusammenarbeit im Verlauf des Projekts verlief. Zunächst 
wird das Team vorgestellt, anschließend wird erläutert, wie die 
Arbeitsaufteilung und Kommunikation erfolgte. Zum Schluss erfolgt eine 
Darstellung der einzelnen Rollen und Aufgaben der Teammitglieder.

\section{Übersicht}
Das Projekt wurde ursprunglich in einer Dreiergruppe gestartet durchgeführt, 
bestehend aus Briel Mohomye, Vanelle Tchinda und Tony Nsangou. Aufgrund 
organisatorischer Schwierigkeiten konnte das dritte Teammitglied jedoch 
nicht offiziell am Projekt teilnehmen. Aus diesem Grund wurde das Projekt 
im weiteren Verlauf von Briel Mohomye und Vanelle Tchinda eigenständig 
durchgeführt. Beide Teammitglieder haben gleichberechtigt an der Konzeption, 
Entwicklung und Dokumentation des Tools mitgewirkt. Ziel war es, 
die Stärken und Kompetenzen beider Personen optimal einzubringen, 
um die Arbeit effizient und zielgerichtet zu gestalten.

\section{Plan zur Zusammenarbeit im Team}
Wir haben Git als Versionskontrollsystem genutzt, um den Code gemeinsam zu
verwalten und Konflikte zu vermeiden. Regelmäßig fanden Meetings mit dem Professor  statt, 
um den Fortschritt zu besprechen und Rückmeldungen einzuholen. Nach jedem 
Meeting mit dem Professor trafen wir uns im Team, um die Aufgaben zu 
verteilen und den weiteren Arbeitsplan zu besprechen.
Jede Person arbeitete eigenständig an ihren zugeteilten Aufgaben.
Dabei gaben wir uns gegenseitig Feedback und konstruktive Kritik,
um die Qualität unserer Arbeit kontinuierlich zu verbessern. 
Dieses Vorgehen haben wir bis zum aktuellen Stand des Projekts beibehalten.

Bezüglich der Implementierung waren ursprünglich zwei Personen für das Backend zuständig,
insbesondere für den Parser-Service und die API-Endpunkte, während eine Person
das Frontend entwickelte. Nachdem eine Person das Team verlassen hatte,
haben die verbleibenden zwei Teammitglieder die Arbeit gerecht untereinander aufgeteilt, 
um alle Aufgaben weiterhin abzudecken.


\section{Funktion der einzelnen Personen und Umsetzungskette}