\chapter{Umsetzung}

\section{Systemarchitektur}

\section{Technologie-Stack}

\section{Implementierung}

\subsection{Parsen der BibTeX-Datei in JSON (Backend)}
Zu Beginn wurde ein Parser-Service implementiert,der eine hochgeladene \texttt{.bib}-Datei einliest 
und deren Inhalte in ein strukturiertes JSON-Format umwandelt.Dies ermöglicht eine einfache Weiterverarbeitung der
Literaturdaten im Frontend. Dabei wurden Bibliotheken
genutzt, um die einzelnen Felder wie Titel, Autoren, Jahr und Publikationsart korrekt zu extrahieren.

\subsection{API-Endpunkte für den Datenaustausch (Backend)}
Für die Kommunikation zwischen Backend und Frontend wurden mehrere REST-API-Endpunkte erstellt.
Diese liefern die geparsten Daten im JSON-Format, nehmen Änderungen vom Frontend entgegen und 
stellen Funktionen zum Speichern und Löschen bereit. Die Endpunkte wurden so gestaltet, dass sie
sowohl einzelne Einträge als auch komplette Datensätze verarbeiten können.

\subsection{Importieren einer BibTeX-Datei (Frontend)}
Im Frontend wurde ein Button \glqq Importieren\grqq{} implementiert,
der es dem Benutzer ermöglicht, eine lokale \texttt{.bib}-Datei auszuwählen.
Nach der Auswahl wird die Datei an das Backend gesendet, dort geparst und die verarbeiteten Daten werden im Frontend angezeigt.

\subsection{Anzeige der Einträge in aufklappbarer Form (Frontend)}
Um die Übersichtlichkeit zu erhöhen, werden die einzelnen Literatur-Einträge in 
einer aufklappbaren Struktur dargestellt. So kann der Benutzer gezielt die Details
eines Eintrags einsehen oder ausblenden.

\subsection{Bearbeiten einzelner Einträge (Frontend)}
Jeder Eintrag kann direkt in der Oberfläche bearbeitet werden.
Änderungen an Feldern wie Titel, Autoren oder Jahr werden sofort in der aktuellen Ansicht
übernommen und können später gespeichert werden.

\subsection{Suchfunktion für Literatur-Einträge (Frontend)}
Eine Suchkomponente ermöglicht das Filtern von Einträgen anhand von Schlagwörtern.
Dies erleichtert das Auffinden bestimmter Publikationen in großen Datenbeständen.

\subsection{Pagination der Einträge (Frontend)}
Um die Navigation bei vielen Einträgen zu vereinfachen, 
wurde eine Paginierung implementiert. Diese teilt die Liste der Einträge in
mehrere Seiten auf und reduziert so die Ladezeit und die visuelle Überlastung.

\subsection{Lokale Speicherung der Daten (Frontend)}
Über den Local Storage des Browsers werden Änderungen lokal gespeichert, sodass diese auch nach 
einem Neuladen der Seite erhalten bleiben. Diese Funktion dient als Zwischenspeicher, 
bevor die Daten endgültig exportiert.

\subsection{Speichern der bearbeiteten Daten (Frontend)}
Der Button \glqq Speichern\grqq{} ermöglicht es, alle vorgenommenen Änderungen dauerhaft zu sichern.
Dabei werden die aktuellen Daten aus dem Local Storage übernommen und entweder im Backend oder als Exportdatei gespeichert.

\subsection{Trennen der Autoreninformationen (Frontend)}
Eine spezielle Logik trennt die Autorenangaben aus der \texttt{.bib}-Datei in einzelne Felder.
Dies erlaubt eine saubere Darstellung und erleichtert die Bearbeitung einzelner Namen.

\subsection{Löschen aller Einträge (Frontend)}
Mit dem Button \glqq Alle löschen\grqq{} können sämtliche Einträge in der aktuellen Ansicht entfernt werden.
Diese Funktion ist besonders nützlich, wenn ein neuer Datensatz importiert werden soll.

\subsection{Exportieren der Daten als XML (Frontend)}
Zum Abschluss bietet der Button \glqq Exportieren\grqq{} die Möglichkeit, 
alle gespeicherten Daten in eine XML-Datei umzuwandeln und herunterzuladen.
Damit können die Daten in anderen Anwendungen weiterverwendet oder archiviert werden.
