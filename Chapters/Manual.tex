\chapter{Manual}
In diesem Kapitel wird die Nutzung des entwickelten Assistenzsystems erläutert. 
Zunächst wird erläutert, wie das Projekt lokal gestartet und lauffähig gemacht 
wird. Im Anschluss folgt eine Schritt-für-Schritt-Anleitung zur Bedienung der 
wichtigsten Funktionen: vom Import einer BibTeX-Datei über die Bearbeitung 
einzelner Einträge bis hin zum Export im OPUS-Format. Das Ziel dieses Kapitels 
besteht darin, zukünftigen Nutzern und Entwicklern eine klare und vollständige 
Anleitung zur Verwendung des Systems zu bieten.


\section{Starten der Projekts}
Um das entwickelte Assistenzsystem auszuführen sind zunächst
die notwendigen Systemvoraussetzung zu erfüllen, die in Tabelle
~\ref{tab:systemvoraussetzungen} dargestellt sind. 

\begin{table}[h]
\centering
\begin{tabular}{|l|p{8cm}|l|}
\hline
\textbf{Komponente} & \textbf{Beschreibung} & \textbf{Beispiel} \\
\hline
Node.js & Laufzeitumgebung für die Ausführung der Anwendung & v24.5.0 \\
\hline
Git & Versionskontrollsystem für den Zugriff auf das Projekt-Repository & v2.50.1 \\
\hline
IDE / Code-Editor & Entwicklungsumgebung zur Bearbeitung und Verwaltung des Quellcodes & Visual Studio Code \\
\hline
Webbrowser & Moderner Browser für die Web Storage API & Chrome \\
\hline
\end{tabular}
\caption{Systemvoraussetzungen für das Assistenzsystem}
\label{tab:systemvoraussetzungen}
\end{table}

\noindent Der Quellcode des Projekts wird über die Versionskontrolle bezogen. Die Installation und der Start erfolgen in folgenden Schritten:

\begin{itemize}
    \item Git-Repository lokal klonen: \texttt{git clone https://github.com/brielmohoms/publiflow}

    \item In das Projektverzeichnis wechseln und Abhängigkeiten installieren: \texttt{npm install}
    
    \item In das Backend verzeichnis wechseln: \texttt{cd backend} und das Projekt starten: \texttt{npm start}

    \item Projekt im Entwicklungsmodus starten: \texttt{npm start}

    \item Die Anwendung dann unter \texttt{http://localhost:3000} im Browser erreichbar.
\end{itemize}


\section{Bedienung des Projekts}
Die entwickelte Anwendung ist bewusst einfach und intuitiv gestaltet, um den Nutzerinnen und Nutzern einen schnellen Einstieg zu ermöglichen.  
Nach dem Start wird die Hauptoberfläche angezeigt, auf der alle zentralen Funktionen unmittelbar zugänglich sind.  
\subsection*{Anleitung zur Bedienung}
Die grundlegende Nutzung des Projekts erfolgt in folgenden Schritten:
\begin{enumerate}
    \item \textbf{Import einer Bib\TeX{}-Datei:}  
    Über den Button \glqq Importieren\grqq{} kann eine lokale \texttt{.bib}-Datei ausgewählt werden.  
    Nach dem Upload werden die Daten automatisch geparst und in der Anwendung angezeigt.  
    
    \item \textbf{Darstellung der Einträge:}  
    Die Literatureinträge erscheinen in einer aufklappbaren Liste. Details können je nach Bedarf ein- oder ausgeblendet werden.  

    \item \textbf{Bearbeitung von Feldern:}  
    Einzelne Felder wie Titel, Autoren , Year ... lassen sich direkt bearbeiten.  
    Änderungen werden sofort in der Ansicht übernommen und können zwischengespeichert werden.  
    Besonderes Augenmerk liegt dabei auf der Trennung der Autoreninformationen in einzelne Felder, um eine präzisere Bearbeitung zu ermöglichen.  

    \item \textbf{Navigation und Suche:}  
    Bei größeren Datensätzen unterstützt eine Paginierung die Übersicht, indem Einträge auf mehrere Seiten verteilt werden.  
    Eine integrierte Suchfunktion erlaubt es, gezielt nach Einträgen zu filtern.  

    \item \textbf{Löschen von Einträgen:}  
    Einträge können einzeln oder gesammelt über die vorgesehenen Lösch-Buttons entfernt werden.  

    \item \textbf{Zwischenspeicherung im Browser:}  
    Um Datenverluste zu vermeiden, werden Änderungen automatisch im Local Storage des Browsers zwischengespeichert.  
    Dadurch bleiben Bearbeitungen auch nach einem Neuladen der Seite erhalten.  

    \item \textbf{Speichern der Änderungen:}  
    Mit dem Button \glqq Speichern\grqq{} können alle Änderungen dauerhaft gesichert werden.  

    \item \textbf{Export der Daten:}  
    Über den \glqq Exportieren\grqq{}-Button lassen sich die bearbeiteten Daten als XML-Datei ausgeben.  
    Diese Datei ist kompatibel mit OPUS und kann dort weiterverwendet werden.  
\end{enumerate}

\noindent
Insgesamt ist die Anwendung so konzipiert, dass alle wesentlichen Funktionen ohne lange Einarbeitung genutzt werden können.  
Die klare Struktur der Oberfläche sowie die direkte Rückmeldung bei Interaktionen gewährleisten, dass die Bedienung auch für neue Nutzerinnen
und Nutzer verständlich und effizient ist.
