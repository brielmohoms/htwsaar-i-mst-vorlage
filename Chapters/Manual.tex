\chapter{Manual}
In diesem Kapitel wird die Nutzung des entwickelten Assistenzsystems erläutert. 
Zunächst wird erläutert, wie das Projekt lokal gestartet und lauffähig gemacht 
wird. Im Anschluss folgt eine Schritt-für-Schritt-Anleitung zur Bedienung der 
wichtigsten Funktionen: vom Import einer BibTeX-Datei über die Bearbeitung 
einzelner Einträge bis hin zum Export im OPUS-Format. Das Ziel dieses Kapitels 
besteht darin, zukünftigen Nutzern und Entwicklern eine klare und vollständige 
Anleitung zur Verwendung des Systems zu bieten.


\section{Starten der Projekts}
Um das entwickelte Assistenzsystem auszuführen sind zunächst
die notwendigen Systemvoraussetzung zu erfüllen, die in Tabelle
~\ref{tab:systemvoraussetzungen} dargestellt sind. 

\begin{table}[h]
\centering
\begin{tabular}{|l|p{8cm}|l|}
\hline
\textbf{Komponente} & \textbf{Beschreibung} & \textbf{Beispiel} \\
\hline
Node.js & Laufzeitumgebung für die Ausführung der Anwendung & v24.5.0 \\
\hline
Git & Versionskontrollsystem für den Zugriff auf das Projekt-Repository & v2.50.1 \\
\hline
IDE / Code-Editor & Entwicklungsumgebung zur Bearbeitung und Verwaltung des Quellcodes & Visual Studio Code \\
\hline
Webbrowser & Moderner Browser für die Web Storage API & Chrome \\
\hline
\end{tabular}
\caption{Systemvoraussetzungen für das Assistenzsystem}
\label{tab:systemvoraussetzungen}
\end{table}

\noindent Der Quellcode des Projekts wird über die Versionskontrolle bezogen. Die Installation und der Start erfolgen in folgenden Schritten:

\begin{itemize}
    \item Git-Repository lokal klonen: \texttt{git clone https://github.com/brielmohoms/publiflow}

    \item In das Projektverzeichnis wechseln und Abhängigkeiten installieren: \texttt{npm install}
    
    \item In das Backend verzeichnis wechseln: \texttt{cd backend} und das Projekt starten: \texttt{npm start}

    \item Projekt im Entwicklungsmodus starten: \texttt{npm start}

    \item Die Anwendung dann unter \texttt{http://localhost:3000} im Browser erreichbar.
\end{itemize}


\section{Bedienung des Projekts}